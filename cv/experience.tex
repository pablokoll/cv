%-------------------------------------------------------------------------------
%	SECTION TITLE
%-------------------------------------------------------------------------------
\cvsection{Experiencia laboral}


%-------------------------------------------------------------------------------
%	CONTENT
%-------------------------------------------------------------------------------
\begin{cventries}

%---------------------------------------------------------
\cventry
    {IT Technical Specialist \& Full Stack Developer} % Job title
    {Tendam} % Organization
    {Madrid, España} % Location
    {Dec. 2024 - Presente} % Date(s)
    {
      \begin{cvitems} % Description(s) of tasks/responsibilities
        \item {Trabajo con Salesforce Commerce B2C utilizando cartridges y pipelines, y tengo experiencia con Salesforce Commerce Cloud (SFCC) y Demandware.}
        \item {Trabajo con metodologías como Agile y Scrum para entregar proyectos a tiempo.}
        \item {Utilización de Jira y Confluence para la gestión de proyectos y documentación.}
      \end{cvitems}
    }
  \vspace{5.5mm}
%---------------------------------------------------------
\cventry
    {Software Engineer \& Full Stack Developer} % Job title
    {Webmefy - Shopify Plus agency} % Organization
    {Madrid, Spain} % Location
    {Jul. 2024 - Nov. 2024} % Date(s)
    {
      \begin{cvitems} % Description(s) of tasks/responsibilities
        \item {Desarrollo Full Stack, implementando cambios en los frontends de Shopify utilizando varios temas basados en Liquid, HTML, CSS y JavaScript.}
        \item {Integración de diferentes sistemas para añadir nuevas funcionalidades a los ecommerces, utilizando las APIs GraphQL, REST y StoreFront de Shopify.}
        \item {Implementación de servicios para clientes como ERPs, CRMs, o varios servicios web para mejorar la funcionalidad.}
        \item {Utilización de Amazon Web Services para alojar y desplegar aplicaciones backend.}
        \item {Trabajo cercano con clientes, manteniendo y entregando nuevos proyectos.}
        \item {Aplicación de metodologías Ágiles y Waterfall, seguimiento de tickets y cronogramas para una gestión eficiente de proyectos.}
      \end{cvitems}
    }
  \vspace{5.5mm}
%---------------------------------------------------------
\cventry
    {Software Engineer \& Backend Developer} % Job title
    {Balloon Group as FastForward} % Organization
    {Buenos Aires, Argentina} % Location
    {Jul. 2022 - Jun. 2024} % Date(s)
    {
      \begin{cvitems} % Description(s) of tasks/responsibilities
        \item {Autogestión de proyectos desde un entendimiento técnico, comprendiendo las necesidades de los clientes y asegurando la calidad del desarrollo.}
        \item {Participación en el equipo backend de desarrollo y mantenimiento de un software para sincronizar el Magento eCommerce multi-instancia de Abbott con su CRM de Salesforce, utilizando tecnologías como Node.js, TypeScript, Express.js, Sequelize, PostgreSQL, Mocha y Docker.}
        \item {Integración del eCommerce con los ERPs y transportistas de los ocho principales países de Abbott en Latinoamérica (Colombia, Perú, Argentina, Brasil, Ecuador, Chile, Costa Rica y México)}
        \item {Creación y mantenimiento de un middleware en NestJS, TypeScript, Prisma, PostgreSQL, Redis, Jest y Docker, implementando funcionalidades para los eCommerces de diversos clientes como Circle K, VickyForm e Indumex.}
        \item {Utilizacion de patrones de diseño: Singleton, Factory Method, Builder, Adapter, Decorator y Facade.}
        \item {Aplicación de metodos de autenticación OAuth2, JWT, API Keys y Token-Based Authentication.}
        \item {Implementación de Git para el trabajo en equipo y la gestión de versiones, diferenciando entre los entornos de producción, desarrollo y testing.}
        \item {Desarrollo e interacción con APIs REST, SOAP y GraphQL para la comunicación entre sistemas.}
        \item {Colaboración en la arquitectura de la infraestructura utilizando Amazon Web Services (AWS) para garantizar la escalabilidad de las aplicaciones.}
        \item {Implementación de pipelines CI/CD en la plataforma de GitLab.}
        \item {Utilización de comunicaciones S/FTP, HTTP/S, SSH y Webhooks.}
        \item {Aplicación de metodologías Agile, Waterfall y Scrum.}
        \item {Utilización de herramientas de gestión de proyectos como Jira, Confluence y Trello.}
      \end{cvitems}
    }
  \vspace{5.5mm}
%---------------------------------------------------------
  \cventry
    {Software Developer \& Full Stack Developer} % Job title
    {Balloon Group as FastForward} % Organization
    {Buenos Aires, Argentina} % Location
    {Nov. 2021 - Jul. 2022} % Date(s)
    {
      \begin{cvitems} % Description(s) of tasks/responsibilities
        \item {Colaboré con el equipo de diseño para crear interfaces de usuario interactivas y responsivas utilizando Adobe XD y Figma.}
        \item {Trabajé en el equipo de mantenimiento y desarrollo del CMS de Toyota, utilizando Laravel y PHP.}
        \item {Desarrollé múltiples componentes en React para los sitios web, mejorando la funcionalidad y la experiencia del usuario.}
        \item {Implementé la comunicación entre el backend y el frontend de la aplicación.}
        \item {Me encargué del mantenimiento de un sitio en WordPress, implementando funcionalidades personalizadas mediante PHP.}
        \item {Aplicación de principios de desarrollo como SOLID, DRY y KISS.}
        \item {Desarrollé la landing page en VueJS de los camiones Hino de Toyota para promocionar los mismos en Argentina.}
        \item { Priorice la implementacion de estrategias efectivas de SEO para aumentar la visibilidad y el tráfico de los sitios web.}
      \end{cvitems}
    }
  \vspace{5.5mm}
%---------------------------------------------------------
  \cventry
    {Full Stack Teaching Assistant} % Job title
    {Henry} % Organization
    {Latam} % Location
    {Ago. 2021 - Oct. 2021} % Date(s)
    {
      \begin{cvitems} % Description(s) of tasks/responsibilities
        \item {Asistente de Bootcamp (TA) para Estudiantes de Desarrollo Full-Stack}
        \item {Coordinar un grupo de estudiantes para lograr su integración en el grupo de estudio.}
        \item{Guiar a los estudiantes durante el curso.}
        \item{Asistir en la resolución de ejercicios y promover la colaboración grupal (Pair Programming).}
        \item{Proponer ideas para mejorar los procesos del Bootcamp.}
        \item{Revisar el repositorio de GitHub.}
      \end{cvitems}
    }

%---------------------------------------------------------
\end{cventries}
